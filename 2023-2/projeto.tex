% Created 2023-09-05 Tue 18:24
% Intended LaTeX compiler: lualatex
\documentclass[11pt]{article}
\usepackage{graphicx}
\usepackage{longtable}
\usepackage{wrapfig}
\usepackage{rotating}
\usepackage[normalem]{ulem}
\usepackage{amsmath}
\usepackage{amssymb}
\usepackage{capt-of}
\usepackage{hyperref}
\usepackage{xurl}
% ======================================================
% =====================  Packages  =====================
% ======================================================
\usepackage[brazil]{babel}
\usepackage[utf8]{inputenc}
\usepackage[a4paper, margin=1in]{geometry}
\usepackage{fancyhdr}
\usepackage[table]{xcolor}
\usepackage{booktabs}
\usepackage{array}
\usepackage{enumitem}
\usepackage{xcolor}
\usepackage{datetime2}
\usepackage[maxbibnames=99, style=numeric, sorting=none]{biblatex}
\addbibresource{pds2.bib}

% ======================================================
% =====================    Info    =====================
% ======================================================
\makeatletter
\DeclareRobustCommand*\course[1]{\gdef\@course{#1}}
\DeclareRobustCommand*\institution[1]{%
  \gdef\@institution{#1}}
\DeclareRobustCommand*\semester[1]{\gdef\@semester{#1}}
\course{Programação e Desenvolvimento de Software II}
\institution{DCC / ICEx / UFMG}
\semester{2023.2}
\newcommand{\thetitle}{\@title{}}
\newcommand{\theauthor}{\@author{}}
\newcommand{\thecourse}{\@course{}}
\newcommand{\theinstitution}{\@institution{}}
\newcommand{\thesemester}{\@semester{}}
\newcommand{\thedate}{\@date{}}
\makeatother
\DTMnewdatestyle{brDateStyle}{%
    \renewcommand{\DTMdisplaydate}[4]{##3/##2/##1}%
    \renewcommand{\DTMDisplaydate}{\DTMdisplaydate}}
\DTMsetdatestyle{brDateStyle}

% ======================================================
% ==================    Page Style    ==================
% ======================================================

\pagestyle{fancy}
\fancyhf{}
\setlength{\headheight}{15pt}
\lhead{\theauthor{} \\ \thecourse{}}
\rhead{\theinstitution{} \\ \thesemester{}}
\rfoot{\thepage}
\hypersetup{
    colorlinks,
    linkcolor={red!50!black},
    citecolor={blue!50!black},
    urlcolor={blue!80!black}
}

\author{Prof. Luigi D. C. Soares}
\date{\today}
\title{Projeto Prático}
\hypersetup{
 pdfauthor={Prof. Luigi D. C. Soares},
 pdftitle={Projeto Prático},
 pdfkeywords={},
 pdfsubject={},
 pdfcreator={Emacs 30.0.50 (Org mode 9.7-pre)}, 
 pdflang={Portuges}}
\begin{document}

\begin{center}
    \Large\bfseries\thetitle{}
\end{center}

\setlist[1]{itemsep=-5pt}
\section{Introdução}
\label{sec:org0cb1c6c}

O grupo deve escolher um problema de seu interesse e realizar todo o
processo de desenvolvimento (análise, projeto e implementação) de uma
solução. Espera-se, ao final, um sistema de porte médio, que utiliza
os conceitos e técnicas vistos durante o curso (modelagem, conceitos
de POO, testes unitários, boas práticas, etc). O programa deve ser
feito baseado na linguagem C++17. Uma lista não exaustiva de \emph{sugestões}
de temas é apresentada abaixo. O tema é aberto à negociação e os
alunos também podem sugerir outros temas de seu interesse.

\begin{itemize}
\item Twitter (tweet, retweet, follow, etc)
\item Gerenciador de tarefas/compromissos (adicionar, remover, listar, histórico, etc)
\item Sistema para biblioteca (busca, reserva, empréstimo, etc)
\item Sistema de gerência de e-commerce
\item Mini rede social (timeline, adicionar/listar amigos, post, etc)
\item UFMG: Carona, Eventos, Bandejão, Gamefication, Oportunidades, \ldots{}
\item Biblioteca de grafos
\item Indexador/organizador de arquivos
\item Outros (Magic, RPG, Reserva de Passagens/Hotel, Netflix, Spotify, Trello/Notion, \ldots{})
\end{itemize}

\textbf{ATENÇÃO: Temas similares poderão ser escolhidos por no máximo 3 grupos!}

\textbf{ATENÇÃO: Dois temas estão ``proibidos'', pois são temas que iremos}
\textbf{utilizar ao longo do semestre. Serão dois (mini) projetos, um sistema}
\textbf{de emails e um sistema para agendamento de consultas. Se o grupo já}
\textbf{estiver escolhido algum destes temas e não tiver nenhuma outra ideia,}
\textbf{poderá ser liberado, desde que haja uma conversa prévia para entender}
\textbf{o que será feito de diferente do que iremos ver em sala.}

O desenvolvimento e a entrega deverão ser feitos utilizando o sistema
de controle de versão GitHub. Sugere-se que commits/pushs sejam feitos
de maneira frequente, por todos os membros do grupo, sempre que houver
alterações.

Lembre-se, o objetivo não é apenas escrever um programa funcional, mas
desenvolver um sistema confiável, reutilizável e de fácil manutenção e
extensão! Logo, tente aplicar todos os conceitos de POO, modularidade
e corretude vistos em sala de aula. Também serão avaliados critérios
como criatividade na solução, assim como a possível implementação de
funcionalidades extras

Durante o desenvolvimento, o grupo deverá utilizar o framework doctest
(\url{https://github.com/onqtam/doctest}) para implementar os testes de
unidade (o mesmo utilizado na sala de aula). Deve haver pelo menos uma
classe de testes para cada uma das principais classes do sistema (por
exemplo, as classes de entidades).

A interface de interação com a aplicação poderá ser feita via terminal
de comando. (A implementação de aplicações gráficas entra como ponto
extra, mas não se sinta pressionado a fazer, tendo em vista que não
faz parte da ementa da disciplina.) Possíveis arquivos necessários
durante a etapa de inicialização do sistema deverão ser fornecidos
pelo grupo.
\section{Modelagem}
\label{sec:orgdcc72df}

As histórias de usuário são uma forma simples de apresentar os requisitos
funcionais desejados para um determinado sistema. São artefatos de
desenvolvimento utilizados principalmente em processos baseados em
metodologias ágeis. As descrições são intencionalmente genéricas,
dando liberdade ao grupo para decidir detalhes da implementação.

Nesta etapa, o grupo deverá identificar possíveis funcionalidades
interessantes de serem incorporadas ao sistema e propor pelo menos
CINCO histórias de usuário. Para cada uma, deverá ser feita uma
descrição sucinta. Tente apresentar histórias de usuário para
diferentes tipos de usuário do sistema (ou considerando papeis
específicos em certos contextos). Cada história de usuário deve
apresentar entre três e cinco critérios de aceitação.

\textbf{Entrega}: o grupo deverá utilizar a ferramenta de \emph{issues} do GitHub para
criação e controle das histórias de usuário. Cada história
corresponderá a uma issue. O título da issue será a história de
usuário, e o corpo da issue deve conter a lista de critérios de
aceitação como checkboxes. Critérios de aceitação muito complexos
talvez devem se tornar uma história por si só (a depender do
julgamento do grupo); é possível converter uma tarefa da lista pelo
próprio GitHub, passando o mouse sobre o item da lista.

Em seguida, faça pelo menos CINCO cartões CRC para as classes que você
julga como sendo as mais importantes. Cada cartão deve
apresentar pelo menos quatro responsabilidades \emph{coesas} (considerando
conhecimento e realização) e mencionar pelo menos duas
colaborações. Você pode usar esse site para lhe auxiliar a fazer o
cartão: \url{https://echeung.me/crcmaker/}
\section{Documentação}
\label{sec:orga328f33}

Durante toda a implementação, quaisquer declarações de
classes e funções (apenas nos arquivos de cabeçalho!) devem ser
documentadas utilizando-se a ferramenta Doxygen
(\url{http://www.doxygen.org/}) e o arquivo README
(\url{https://docs.github.com/pt/repositories/managing-your-repositorys-settings-and-features/customizing-your-repository/about-readmes}).
A sintaxe dos comentários pode ser encontrada nas notas de aula (os
comentários com três barras, seguido de notações especiais como
@brief, @param, e @return).

Além disso, o grupo deve utilizar o arquivo README.md do GitHub para
documentar o projeto em si. O arquivo README.md deve conter (pelo
menos!):

\begin{itemize}
\item Uma breve apresentação do projeto/problema
\item Menções a quaisquer dependências que sejam necessárias para
o correto funcionamento
\item Instruções de compilação e execução (incluindo, caso aplicável, a
necessidade de um sistema operacional específico; se possível, para
facilitar a correção, peço que garantam o funcionamento no ambiente
linux/wsl)
\end{itemize}
\section{Comentários Gerais}
\label{sec:org1370881}

\begin{itemize}
\item O trabalho deverá ser feito em grupos com quatro ou cinco alunos
\item Trabalhos copiados serão, obviamente, zerados!!!
\item Na entrega final, será considerado o último commit na branch principal do projeto
\item O arquivo deve conter um arquivo Makefile com as opções make e make run.
\end{itemize}
\section{Critérios de Avaliação}
\label{sec:orgad8dce4}

\subsection{Entrega Parcial (Modelagem)}
\label{sec:org3f1337d}

\subsubsection{Histórias de Usuário (3 pontos):}
\label{sec:org6cee8c5}
\begin{itemize}
\item \(-0.6\) pontos por cada história não entregue (considerando o mínimo de cinco)
\item \(-0.1\) ponto por cada critério de aceitação não entregue,
incoerente com a história, ou superficial (mínimo três por história)
\item \(-0.3\) pontos por cada história muito simples / pouco expressiva
\item Em caso de mais de cinco histórias, a nota corresponderá as cinco
mais bem availadas
\end{itemize}
\subsubsection{Cartões CRC (3 pontos):}
\label{sec:org3ac1486}
\begin{itemize}
\item \(-0.6\) pontos por cada cartão CRC não entregue (considerando o mínimo de cinco)
\item \(-0.1\) ponto por cada responsabilidade não entregue (mínimo  quatro por cartão)
\item \(-0.1\) ponto por cada colaboração não entregue (mínimo dois por cartão)
\item Em caso de mais de cinco cartões, a nota corresponderá aos cinco
mais bem avaliados
\end{itemize}
\subsection{Entrega Final (Implementação)}
\label{sec:orga9422b7}

\subsubsection{Documentação e Estilo (4 pontos):}
\label{sec:org9f00ffe}
\begin{itemize}
\item \(-2\) pontos se README incompleto/pouco detalhado
\item \(-2\) pontos se não utiizou/utilizou incorretamente o Doxygen
\item \(-1\) ponto se descrição das classes/funções pouco detalhadas
\item \(-1\) ponto se nomes de atributos e funções não padronizados
\item \(-1\) ponto se indentação não padronizada
\end{itemize}
\subsubsection{Funcionamento (6 pontos):}
\label{sec:org660f59b}
\begin{itemize}
\item \(-6\) pontos se sequer compila
\item \(-4\) pontos se compila, mas não executa
\item \(-0.5\) \textasciitilde{} \(1\) ponto por cada erro durante a execução, a depender do erro
\end{itemize}
\subsubsection{Boas Práticas e POO (6 pontos):}
\label{sec:orgfa87fd2}
\begin{itemize}
\item \(-1\) pontos se não utilizou o conceito de encapsulamento corretamente
\item \(-1\) pontos se não utilizou composição/herança/interfaces corretamente
\item \(-1\) pontos se não utilizou polimorfismo corretamente
\item \(-1\) ponto se não modularizou o código (arquivos hpp e cpp)
\item \(-1\) ponto se não modularizou o projeto (diferentes diretórios)
\item \(-2\) pontos se não criou o Makefile
\end{itemize}
\subsubsection{Programação Defensiva / Tratamento de Exceções (4 pontos)}
\label{sec:org8e50124}
\begin{itemize}
\item \(-4\) pontos se não fez nenhum tratamento/sanitização das entradas
\item \(-2\) pontos se não fez tratamento de exceções corretamente
\end{itemize}
\subsubsection{Testes de Unidade (4 pontos)}
\label{sec:org67b3027}
\begin{itemize}
\item \(-4\) pontos se não fez nenhum teste de unidade
\item \(-2\) pontos se faltaram testes para classes importantes/principais
\end{itemize}
\subsubsection{Criatividade (2 pontos extras)}
\label{sec:org472588a}
\subsection{Nota Final}
\label{sec:orgcae9fcf}

Será avaliada a partipação individual de cada membro do grupo, como um
valor entre \(0\) e \(1\), baseado na proporção de commits no GitHub e
interação durante as aulas de acompanhemento do projeto, por email,
mensagens, etc. Sendo \(P\) a nota de participação e \(N_i\) a nota de cada
item descrito acima, a nota final será dada por

\[\mathit{NotaFinal} = P \sum_i N_i.\]
\end{document}